\section{Normalisasjon}
\subsection*{Hvorfor gjør man normalisasjon?}
\begin{frame}{Hvorfor normalisasjon?}
\begin{itemize}[<+->]
    \item Unngå redundante informasjoner
    \item Bedre å oppdatere informasjoner hvis de bare finnes en gang
    \item Lettere å bruke en stor database
    \item Visualisering av en database
\end{itemize}
\end{frame}

\subsection*{Viktige begrep}
\begin{frame}{Begrep}
\begin{itemize}[<+->]
    \item \textbf{Entitet: }En tabell i databasen
    \item \textbf{Attributt: }En spalte i en tabell
    \item \textbf{Nøkkel: }En spalte som bestemmer verdier i en/noen flere spalter
    \item \textbf{Supernøkkel: }En nøkkel som bestemmer verdier i alle andre spaltene
    \item \textbf{Kandidatnøkkel: }En minimal supernøkkel
\end{itemize}
\end{frame}

\begin{frame}{Functional Dependencies}
\begin{itemize}[<+->]
    \item En eller flere spalter bestemmer verdier av andre spalter
    \item $A \rightarrow B$ (A bestemmer B)
    \item $A,B \rightarrow C, D, E$ (A og B bestemmer C, D og E)
    \item PersonNr $\rightarrow$ Navn (det samme nummer er det samme person)
    \item ZipCode $\rightarrow$ City (den samme postkoden er den samme byen)
\end{itemize}
\end{frame}

\subsection*{Anomalier}
\begin{frame}{Eksempel}
%\begin{center}
\begin{tabular}{l|l|l|l|l}
 StudentNr & Name & Address & KursNr & KursName \\\hline
 580 & Ola NordmaNN & 5075 Bergen Fv14 & INF237 & Algoritme Engineering\\
 580 & Ola NordmaNN & 5075 Bergen Fv14 & INF273 & Meta Heuristikker\\
 580 & Ola NordmaNN & 5075 Bergen Fv14 & INF227 & Logik\\
 256 & Max MustermaNN & 5055 Bergen Lv85 & INF237 & Algoritme Engineering\\
\end{tabular}
\\[5mm]
Hva er problemet her?\\\pause
Informasjoner blir lagret dobbelt.\\
Hvis kursnavn eller studentadressen endrer seg, må den endres flere ganger.
%\end{center}
\end{frame}

\begin{frame}{Anomalier}
\begin{itemize}[<+->]
    \item \textbf{Insertion Anomaly:} Den samme informasjonen blir lagt inn med andre verdier
    \item \textbf{Update Anomaly:} Dobbel informasjon blir ikke oppdatert overalt
    \item \textbf{Deletion Anomaly}: Data forsvinner fordi den ikke finnes i en egen tabell
\end{itemize}

\pause

\medskip

Eksempler:
\begin{itemize}[<+->]
    \item \textbf{Insertion Anomaly:} Studenten tar et kurs til, men har nå en annen adresse 
    \item \textbf{Update Anomaly:} Kursnavnet blir oppdatert, men ikke for alle studenter
    \item \textbf{Deletion Anomaly:} En student melder seg ut av alle kurser, så forsvinner han i hele databasen
\end{itemize}
\end{frame}

\begin{frame}{Hvordan løser man problemene?}
\begin{tabular}{l|l|l}
 StudentNr & Name & Address\\\hline
 580 & Ola NordmaNN & 5075 Bergen Fv 14\\
 256 & Max MustermaNN & 5055 Bergen Lv 85\\
\end{tabular}
\vfill
\begin{tabular}{l|l}
KursNr & KursName \\\hline
INF237 & Algorithm Engineering\\
INF273 & Meta Heuristikker\\
INF227 & Logikk\\
\end{tabular}
\hfill
\begin{tabular}{l|l}
 StudentNr & KursNr\\\hline
 580 & INF237\\
 580 & INF273\\
 580 & INF227\\
 256 & INF237\\
\end{tabular}
\end{frame}

\subsection*{Normalformer}
\begin{frame}{1. Normalform (1NF)}
    \begin{block}{1. Normalform}
    Hvert attributt (spalte) må være atomar, den kan ikke har flere verdier (lister etc. er forbud).
    \end{block}
    \vfill
    \pause
    \begin{tabular}{l|l}
     StudentNr & KursNr\\\hline
     580 & [INF237, INF273, INF227]\\
     256 & INF237\\
     \end{tabular}
     \hfill
     \pause
     $\rightarrow$
     \hfill
     \begin{tabular}{l|l}
     StudentNr & KursNr\\\hline
     580 & INF237\\
     580 & INF273\\
     580 & INF227\\
     256 & INF237\\
    \end{tabular}
\end{frame}

\begin{frame}{2. Normalform (2NF)}
    \begin{block}{2. Normalform}
    Alle ikke-nøkkel-attributter er fullstendig funksjonell avhengig av en primærnøkkel.\\
    En ikke primær-attributt kan ikke være avhengig av bare en del av nøkkelen.
    \end{block}
    \vfill
    \only<1>{\begin{tabular}{l|l|l}
     StudentNr & KursNr & StudentNavn\\\hline
     580 & INF237 & Ola\\
     580 & INF273 & Ola\\
     580 & INF227 & Ola\\
     256 & INF237 & Max\\
     \end{tabular}
     }
     \only<2>{
     \begin{tabular}{l|l}
     StudentNr & KursNr\\\hline
     580 & INF237\\
     580 & INF273\\
     580 & INF227\\
     256 & INF237\\
    \end{tabular}
    \hfill
    \begin{tabular}{l|l}
     StudentNr & StudentNavn\\\hline
     580 & Ola\\
     580 & Ola\\
     580 & Ola\\
     256 & Max\\
    \end{tabular}
    }
\end{frame}

\begin{frame}{3. Normalform (3NF)}
\begin{block}{3. Normalform}
    Det finnes ingen transitive avhengigheter.\\
    Det finnes ikke tre attributter A, B, C der $A \rightarrow B$ og $B \rightarrow C$ gjelder.
    \end{block}
    \vfill
    \only<1>{\begin{tabular}{l|l|l}
     Universitetet & By & Land\\\hline
     UiB & Bergen & Norge\\
     HUB & Berlin & Tyskland\\
     NTNU & Trondhjem & Norge\\
     \end{tabular}
     }
     \only<2>{
     \begin{tabular}{l|l}
     Universitetet & By\\\hline
     UiB & Bergen\\
     HUB & Berlin\\
     NTNU & Trondhjem\\
    \end{tabular}
    \hfill
    \begin{tabular}{l|l}
     By & Land\\\hline
     Bergen & Norge\\
     Berlin & Tyskland\\
     Trondhjem & Norge\\
    \end{tabular}
    }
\end{frame}

\begin{frame}{Boyce–Codd Normalform (BCNF)}
\begin{block}{Boyce-Codd Normalform ($3 \frac{1}{2}$NF)}
    Det finnes ikke to overlappende kandidatnøkler. Skjer nesten aldri.
    \end{block}
    \vfill
    $Tabell(A, B, C, D)$\\
    $A,B\rightarrow C,D; $ $C\rightarrow B$\medskip
    
    \pause
    Løsning: Split opp tabellen for å fjerne syklisk avhengig:\\
    $Tabell1(A, C, D)$, $Tabell2(C, B)$\\
    $A\rightarrow C,D; $ $C\rightarrow B$\medskip
\end{frame}

\subsection*{Spørretid}
\begin{frame}{Spørsmål?}
    \begin{figure}
        \centering
        \includegraphics[height = 4.9cm]{images/guillaume9.jpg}
        \caption{Guillaume foran Tvindefossen}
        \label{fig:guillaume9}
    \end{figure}
\end{frame}